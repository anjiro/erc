\documentclass[11pt]{article}
\newcommand{\firstname}{Firstname}
\newcommand{\surname}{Surname}
\newcommand{\acronym}{ACRONYM}

\usepackage[part=B2,firstname=\firstname,surname=\surname,acronym=\acronym]{erc}

\begin{document}

  \begin{center}
    \vbox{\vspace{1.5cm}}
    \Large{\textbf{%
        ERC Starting Grant 2016\\
        Research proposal [Part B2]\\}
    }
    \vspace{1cm}
  \end{center}

  \note{15 pages\\
    Description of the scientific and technical aspects of the
    project, demonstrating the ground-breaking nature of the research,
    its potential impact and research methodology. The proposal will
    also need to clearly specify the percentage of the applicant's
    total working time that will be spent in the EU or an Associated
    Country and the percentage of the applicant’s total working time
    that will be devoted to the project, as well as a full estimation
    of the real project cost.
  }

  \tableofcontents

  \section{State-of-the-art and objectives}

     \note{Specify the proposal objectives in the context of the state
     of the art in the research field. It should be clear how
     and why the proposed work is important for the field, and
     what impact it will have if successful, such as how it may
     open up new horizons or opportunities for science, technology or
     scholarship. Specify any particularly challenging or unconventional
     aspects of the proposal, including multi- or inter-disciplinary
     aspects.}

  \section{Methodology}

    \note{Describe the proposed methodology in detail including any key
    intermediate goals. Explain and justify the methodology in relation
    to the state of the art, and particularly novel or unconventional
    aspects addressing the 'high-risk/high-gain' balance. Highlight any
    intermediate stages where results may require adjustments to the
    project planning. In case you ask that team members are engaged by
    another host institution their participation has to be fully
    justified by the scientific added value they b ring to the project.}

  \section{Resources}

    \note{\textbf{It is strongly recommended to use the Part B2 budget table
    template} to facilitate the assessment of resources by the panels.
    State the amount of funding considered necessary to fulfil t he
    research objectives. The project cost estimation should be as
    accurate as possible. The requested budget should be fully
    justified and in proportion to the actual needs. The evaluation
    panels assess the estimated costs carefully; \textbf{unjustified
    budgets will be consequently reduced}. Specify your commitment to
    the project and how much time you are willing to devote to the
    proposed project (You are expected to spend as a minimum 50\% for
    STG and 40\% for COG of your working time on the ERC project).
    Describe the size and nature of the team, indicating, where
    appropriate, the key team members and their roles. The
    participation of team members engaged by other host institutions
    should be justified and in relation to the additional financial
    cost this may impose. When estimating your personnel costs take
    into account the dedicated working time to run the project.
    Specify any existing resources that will contribute to the
    project. Describe other necessary resources, such as
    infrastructure and equipment. Include a short technical
    description of any requested equipment, why you need it and how
    much you plan to use it for the project. Please include a
    realistic estimation of the costs for Open Access to project
    outputs. Costs for providing immediate Open Access to publications
    (article processing charges/book processing charges) are eligible
    if they are incurred during the lifetime of the project.}

  \bibliographystyle{plainnat}

\end{document}
